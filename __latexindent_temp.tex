\chapter{Introdução} \label{ch:intro}

	O ensino de Engenharia no Brasil se inicia formalmente em 1792, com a criação da Real Academia de Artilharia, Fortificação e Desenho, na cidade do Rio de Janeiro. Esta foi pioneira entre instituições do seu tipo na América Latina e foi precursora da Escola Politécnica da Universidade Federal do Rio de Janeiro (Poli\abbrev{Poli}{Escola Politécnica}/UFRJ\abbrev{UFRJ}{Universidade Federal do Rio de Janeiro}) e também do Instituto Militar de Engenharia.
	
	A regulamentação nacional da profissão de Engenheiro só se dá em 1933, com a criação do Conselho Federal de Engenharia e Arquitetura e dos respectivos Conselhos Regionais \cite{historia-confea}.'
	
	Neste contexto uma das características a ser repensada é a metodologia tradicional de ensino aplicada na maioria das Universidades, focada em um modelo expositivo e centrado na figura do instrutor. Esta apresenta deficiências objetivas na capacidade de 

	\section{Motivação}

		Blablabla
		
	\section{Objetivos}
	
		Este trabalho faz a análise das possibilidades de implementação da metodologia de \textit{Problem Based Learning} no curso de Engenharia Elétrica da Poli/UFRJ.
		
	\section{Estrutura}
	
		Nesta seção é apresentada a estrutura deste projeto e um breve resumo de cada um dos capítulos. Este trabalho está dividido em 5 capítulos:
		
		O \textbf{Capítulo \ref{ch:revisao}} apresenta uma revisão bibliográfica d
		
		O \textbf{Capítulo \ref{ch:metodo}} mostra o método aplicado na proposta de implementação da metodologia.
		
		O \textbf{Capítulo \ref{ch:resultados}} faz uma análise dos resultados esperados 
		
		O \textbf{Capítulo \ref{ch:conclusoes}} apresenta as conclusões e as propostas de projetos futuros.