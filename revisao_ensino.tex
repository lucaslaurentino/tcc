\chapter{Como Engenheiros Aprendem} \label{ch:revisao_ensino}

	Para que se possa entender as estruturas que fundamentam metodologias de ensino, o porquê e como elas funcionam, é preciso fazer uma reflexão mais profunda sobre as formas como pessoas, e engenheiros especificamente, adquirem conhecimento. Considerando-se a limitação objetiva de que este tema por si só é de muito mais abrangência e profundidade que o escopo deste trabalho, este capítulo faz uma breve revisão das principais teorias sobre ele.

	\section{Levantamento Histórico}
	
		Até meados do século XIX, a teoria do Behaviorismo era amplamente aceita dentro da academia, estruturando-se sobre algumas premissas básicas. Entre 
	
		Já na segunda metade do século XIX, começa a teoria do Construtivismo surge em oposição ao Behaviorismo
		
		Esta é a premissa que fundamenta os capítulos restantes deste trabalho.

	\section{Aprendizagem Baseada em Problemas}

		A metodologia da Aprendizagem base
