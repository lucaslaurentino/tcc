\chapter{Introdução} \label{ch:intro}



	\section{Motivação}
	
		A Escola Politécnica \abbrev{Poli}{Escola Politécnica} da Universidade Federal do Rio de Janeiro \abbrev{UFRJ}{Universidade Federal do Rio de Janeiro} (Poli/UFRJ) foi o berço do ensino de Engenharia no Brasil. 
		
	\section{Objetivos}
	
		Este trabalho faz a análise das possibilidades de implementação da metodologia de \textit{Problem Based Learning} no curso de Engenharia Elétrica da Poli/UFRJ.
		
	\section{Metodologia}
		
	\section{Estrutura deste Trabalho}
	
		Nesta seção é apresentada a estrutura deste projeto e um breve resumo de cada um dos capítulos. Este trabalho está dividido em 5 capítulos:
		
		O \textbf{Capítulo \ref{ch:revisao}} apresenta uma revisão bibliográfica d
		
		O \textbf{Capítulo \ref{ch:metodo}} mostra o método aplicado na proposta de implementação da metodologia.
		
		O \textbf{Capítulo \ref{ch:resultados}} faz uma análise dos resultados esperados 
		
		O \textbf{Capítulo \ref{ch:conclusoes}} apresenta as conclusões e as propostas de projetos futuros.
		