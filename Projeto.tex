\documentclass[grad,numbers]{coppe}
\usepackage{amsmath,amssymb}
\usepackage{hyperref}
\usepackage[utf8]{inputenc}
\usepackage[brazil]{babel}
\usepackage[T1]{fontenc}
\usepackage{graphicx}
\usepackage{indentfirst}

\graphicspath{{Figuras/}}

\makelosymbols
\makeloabbreviations

\begin{document}
	\title{Título do projeto}
	\foreigntitle{Project title}
    %\title{Implementação de \textit{Problem Based Learning} no curso de Engenharia Elétrica da Universidade Federal do Rio de Janeiro}
    %\foreigntitle{Implementation of Problem Based Learning in the Electrical Engineering course at UFRJ}
    \author{Lucas Laurentino}{da Costa Ribeiro}
    \advisor{Prof.}{Walter}{Issamu Suemitsu}{D.Sc.}
    
    \examiner{Prof.}{Nome do Primeiro Examinador Sobrenome}{D.Sc.}
    \examiner{Prof.}{Nome do Segundo Examinador Sobrenome}{Ph.D.}
    
    \department{EET}
    
    \date{09}{2019}
    
    \keyword{Ensino de Engenharia}
    \keyword{\textit{Problem Based Learning}}
    
    \maketitle
    
    \frontmatter

\dedication{
	%Ao meu avô, \\Luiz Laurentino da Costa
}

\chapter*{Agradecimentos}

  %Gostaria de agradecer a todos.

\begin{abstract}

  Apresenta-se, nesta tese, ...

\end{abstract}

\begin{foreignabstract}

In this work, we present ...

\end{foreignabstract}

\tableofcontents
\listoffigures
\listoftables
\printlosymbols
\printloabbreviations

% Para atualizar as listas de abreviaturas e símbolos, executar os seguintes códigos:
% makeindex -s coppe.ist -o Projeto.lab Projeto.abx
% makeindex -s coppe.ist -o Projeto.los Projeto.syx

\mainmatter

\chapter{Introdução}

	A Escola Politécnica \abbrev{Poli}{Escola Politécnica} da Universidade Federal do Rio de Janeiro \abbrev{UFRJ}{Universidade Federal do Rio de Janeiro} (Poli/UFRJ) foi o berço do ensino de Engenharia no Brasil. 

	\section{Motivação}
		\abbrev{Poli}{Escola Politécnica}
		
	\section{Objetivos}
	
		Este trabalho faz a análise das possibilidades de implementação da metodologia de \textit{Problem Based Learning} no curso de Engenharia Elétrica da Poli/UFRJ.
		
	\section{Metodologia}
		
	\section{Estrutura deste Trabalho} 

\chapter{Revisão Bibliográfica}

	\section{\textit{Problem Based Learning}}

  
\chapter{Método Proposto} \label{ch:metodo}

\chapter{Resultados Esperados e Discussão} \label{ch:resultados}  
  
\chapter{Conclusões} \label{ch:conclusoes}

	\section{Trabalhos Futuros}


\backmatter
\bibliographystyle{coppe-unsrt}
\bibliography{Projeto}

\appendix
\chapter{Algumas Demonstrações}
  	
\end{document}
