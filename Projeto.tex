\documentclass[grad,numbers]{coppe}
\usepackage{amsmath,amssymb}
\usepackage{hyperref}
\usepackage[utf8]{inputenc}
\usepackage[brazil]{babel}
\usepackage[T1]{fontenc}
\usepackage{graphicx}

\graphicspath{{Figuras/}}

\makelosymbols
\makeloabbreviations

\begin{document}
    \title{Implementação de \textit{Problem Based Learning} no curso de Engenharia Elétrica da Universidade Federal do Rio de Janeiro}
    \foreigntitle{Implementation of Problem Based Learning in the Electrical Engineering course at UFRJ}
    \author{Lucas Laurentino}{da Costa Ribeiro}
    \advisor{Prof.}{Walter}{Issamu Suemitsu}{D.Sc.}
    
    \examiner{Prof.}{Nome do Primeiro Examinador Sobrenome}{D.Sc.}
    \examiner{Prof.}{Nome do Segundo Examinador Sobrenome}{Ph.D.}
    \examiner{Prof.}{Nome do Terceiro Examinador Sobrenome}{D.Sc.}
    
    
    \department{EET}
    
    \date{08}{2019}
    
    \keyword{Primeira palavra-chave}
    \keyword{Segunda palavra-chave}
    \keyword{Terceira palavra-chave}
    
    \maketitle
    
    \frontmatter

\dedication{A alguém cujo valor é digno desta dedicatória.}

\chapter*{Agradecimentos}

  Gostaria de agradecer a todos.

\begin{abstract}

  Apresenta-se, nesta tese, ...

\end{abstract}

\begin{foreignabstract}

In this work, we present ...

\end{foreignabstract}

\tableofcontents
\listoffigures
\listoftables
\printlosymbols
\printloabbreviations

\mainmatter
\chapter{Introdução}


\chapter{Revisão Bibliográfica}

  
\chapter{Método Proposto}
  
  
\chapter{Resultados e Discussões}
  
  
\chapter{Conclusões}
  

\backmatter
\bibliographystyle{coppe-unsrt}
\bibliography{example}

\appendix
\chapter{Algumas Demonstrações}
  	
\end{document}
